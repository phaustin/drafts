%%%%%%%%%%%%%%%%%%%%%%%%%%%%%%%%%%%%%%%%%%%%%%%%%%%%%%%%%%%%%%%%%%%%%
% PREAMBLE
%%%%%%%%%%%%%%%%%%%%%%%%%%%%%%%%%%%%%%%%%%%%%%%%%%%%%%%%%%%%%%%%%%%%%
%
% The following two commands will generate a PDF that follows all the 
% requirements for submission and peer review.  Uncomment these commands to 
% generate this output (and comment out the two lines below.)
%
% DOUBLE SPACE VERSION FOR SUBMISSION TO THE AMS
\documentclass[12pt]{article}
\usepackage{ametsoc}
%
% The following two commands will generate a single space, double column paper 
% that closely matches an AMS journal page.  Uncomment these commands to 
% generate this output (and comment out the two lines above. FOR AUTHOR USE 
% ONLY. PAPERS SUBMITTED IN THIS FORMAT WILL BE RETURNED TO THE AUTHOR for 
% submission with the correct formatting.
%
% TWO COLUMN JOURNAL PAGE LAYOUT FOR AUTHOR USE ONLY
%%%%\documentclass[10pt]{article}
%%%%\usepackage{ametsoc2col}
%
%%%%%%%%%%%%%%%%%%%%%%%%%%%%%%%%%%%%%%%%%%%%%%%%%%%%%%%%%%%%%%%%%%%%%
% ABSTRACT
%
% Enter your Abstract here
%%%%%%%%%%%%%%%%%%%%%%%%%%%%%%%%%%%%%%%%%%%%%%%%%%%%%%%%%%%%%%%%%%%%%
\newcommand{\myabstract}{
A technique for the tracking of individual clouds in an large eddy 
simulation model is presented.
}
%
\begin{document}
%
%%%%%%%%%%%%%%%%%%%%%%%%%%%%%%%%%%%%%%%%%%%%%%%%%%%%%%%%%%%%%%%%%%%%%
% TITLE
%
% Enter your TITLE here
%%%%%%%%%%%%%%%%%%%%%%%%%%%%%%%%%%%%%%%%%%%%%%%%%%%%%%%%%%%%%%%%%%%%%
\title{\textbf{\large{Something about cloud tracking}}}
%
% Author names, with corresponding author information. 
% [Update and move the \thanks{...} block as appropriate.]
%
\author{\textsc{Jordan T. Dawe}
	\thanks{\textit{Corresponding author address:} 
	Jordan T. Dawe, 
        Department of Earth and Ocean Sciences, 
        University of British Columbia, 
	6339 Stores Road, 
        Vancouver, BC, 
        V6T 1Z4. 
	\newline{E-mail: jdawe@eos.ubc.ca}}\quad\textsc{and Philip Austin}\\
\textit{\footnotesize{Department of Earth and Ocean Sciences, 
                      University of British Columbia, Vancouver, BC}}
}
%
% Formatting done here...Authors should skip over this.  See above for abstract.
\ifthenelse{\boolean{dc}}
{
\twocolumn[
\begin{@twocolumnfalse}
\amstitle

% Start Abstract (Enter your Abstract above.  Do not enter any text here)
\begin{center}
\begin{minipage}{13.0cm}
\begin{abstract}
	\myabstract
	\newline
	\begin{center}
		\rule{38mm}{0.2mm}
	\end{center}
\end{abstract}
\end{minipage}
\end{center}
\end{@twocolumnfalse}
]
}
{
\amstitle
\begin{abstract}
\myabstract
\end{abstract}
}
%%%%%%%%%%%%%%%%%%%%%%%%%%%%%%%%%%%%%%%%%%%%%%%%%%%%%%%%%%%%%%%%%%%%%
% MAIN BODY OF PAPER
%%%%%%%%%%%%%%%%%%%%%%%%%%%%%%%%%%%%%%%%%%%%%%%%%%%%%%%%%%%%%%%%%%%%%

\section{Introduction}

%==============================================================================

\section{Cloud Tracking}

* Label all cloud core points
* Find depth of point from surface of core
* Find deepest core points, assign to cloudlets
* Iteratively expand into next deepest points
* Any points at that depth that are not in a cloudlet are assigned to new 
    cloudlets
* This allows clouds to merge and split
* Once all the core points are selected, the shell points are joined into 
    cloudlets as well.

Then the cloudlets are joined into clouds.  There are several ways to do this,
and the exact method used can be tailored to the problem beign studied.

At the first time step, cloudlets with touching core are joined into clouds

At subsequent timesteps
 * cloudlets that overlap with a cloud at a previous timestep are assigned the same cloud number
 * then cloudlets with cores that touch the other clouds are joined

Then from the graph formed, seperate out clusters.

%==============================================================================

\section{Tracked Cloud Statistics}







\begin{equation}
\label{eq:E_minus_D} 
E - D = \int_C \rho ( \mathbf{u} -  \mathbf{u_i}) \cdot d\mathbf{C},
\end{equation}

\subsection{Cloud Surface Interpolation}

\ref{fig:direct_vs_tracer}).

%==============================================================================

\section{Conclusions}


%==============================================================================


\begin{acknowledgment}
Support for this research was provided by the Canadian Foundation for Climate 
and Atmospheric Science through the Cloud Aerosol Feedback and Climate 
network.  Figures were generated using the matplotlib library in the Python
programming language.
\end{acknowledgment}

% Use appendix}[A], {appendix}[B], etc. etc. in place of appendix 
% if you have multiple appendixes.
%\ifthenelse{\boolean{dc}}
%{}
%{\clearpage}
%\begin{appendix}
%\section*{\begin{center}Appendix Title Is Entered Here (Primary heading)\end{center}}
%\subsection{First appendix secondary heading}

%\subsection{Second appendix secondary heading}

%\subsubsection{First appendix tertiary heading}

%\subsubsection{Second appendix tertiary heading}

%\paragraph{First appendix quaternary heading}

%\paragraph{Second appendix quaternary heading}

%\end{appendix}

% Create a bibliography directory and place your .bib file there.
\ifthenelse{\boolean{dc}}
{}
{\clearpage}
\bibliographystyle{./ametsoc}
\bibliography{./bibliography/cloud_tracking}

%%%%%%%%%%%%%%%%%%%%%%%%%%%%%%%%%%%%%%%%%%%%%%%%%%%%%%%%%%%%%%%%%%%%%
% FIGURES
%%%%%%%%%%%%%%%%%%%%%%%%%%%%%%%%%%%%%%%%%%%%%%%%%%%%%%%%%%%%%%%%%%%%%

\begin{figure}[t]
  \noindent\includegraphics[width=40pc,angle=0]{./figures/figure1}\\
  \caption{Schematic representation of our cloudlet algorithm.}\label{fig:figure1}
\end{figure}

\begin{figure}[t]
  \noindent\includegraphics[width=40pc,angle=0]{./figures/figure1}\\
  \caption{Associating cloudlets with cloud from previous time step.}\label{fig:figure1}
\end{figure}


\end{document}
