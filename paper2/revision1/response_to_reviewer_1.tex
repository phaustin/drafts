\documentclass[12pt]{article}
\begin{document}
\textbf{Section 3.1. This description of the cloud-tracking algorithm is 
difficult to follow. What distinguishes the different yellow areas that are 
otherwise adjacent in Figure 1? How is cloud that is connected to more than one 
core associated with only one core?}

\vspace{5mm}

We have re-written Section 3.1 extensively to try to make it clearer.

\vspace{5mm}

%-----------------------------------

\textbf{p. 23243, lines 5-7. Are these inter-cloud or intra-cloud 
correlations? Does this include clouds of all heights?}

\vspace{5mm}

These are inter-cloud correlations, done upon all clouds that reach a given 
height.  Clouds that are not present at a given height are not included in the 
cross-correlations, which is why the significant correlation level changes with
height. We have modified lines 5-7 to read ``Cross-correlations between the 
horizontal mean properties of all clouds present at a given height reveal 
strong relationships between the mean cloud properties.''

\vspace{5mm}

%-----------------------------------

\textbf{p. 23243, line 14. If $a$ and $M$ have near-unity correlations, this 
implies that variations in $w$ can be neglected. Should this sentence read
``characterized by two variables: $\theta_\rho$ and $a$''?}

\vspace{5mm}

We do not agree that near-unity correlations between $a$ and $M$ implies that
variations in $w$ can be neglected. $w$ has variability independent of 
$\theta_\rho$ and $a$ which may be neccessary to characterize the behaviour of 
the BOMEX cloud field.

\vspace{5mm}

%-----------------------------------

\textbf{p. 23243, lines 26-28. Please elaborate on how this can be seen or
quantified from Figure 8. Also, how negative is the anticorrelation between 
$\theta_l$ and $\theta_\rho$ at 500 meters in the cutoff portion of Figure 
7?}

\vspace{5mm}

Upon reflection, we realise that this statement is unjustified, and we have 
removed it from the paper.  At 500 meters, the anticorrelation between 
$\theta_l$ and $\theta_\rho$ is $\approx$ -0.9.  We have added this 
information to the text.

\vspace{5mm}

%-----------------------------------  

\textbf{p. 23244, line 4. Here and some other places, should this be Romps and
Kuang?}

\vspace{5mm}

Yes, this should be Romps and Kuang. We have corrected this error.

\vspace{5mm}

%-----------------------------------

\textbf{Figure 9. This figure is described inadequately. Since the dots in b 
are the same as in a, but connected by lines differently, why are they shown 
twice? In c, should these lines be labeled by the level? What correlation is 
being shown in d?}

\vspace{5mm}

In an attempt to make Figure 9 clearer, we have removed the dots from (a) and 
removed the lines from (b), added some level labels and highlighted lines every
200 m in c), and removed every second level from the plots.  We have changed 
the figure caption to read:

``Method used to determine correlations between lower- and upper-level 
cloud properties. a) Numerical particles are released once per minute from an 
initial level in the cloud and advected vertically with the mean vertical 
velocity of the cloud until the particle leaves the cloud. (Lines show the 
time-height trajectories of the numerical particles and colours show the 
cloud's vertical velocity.) b) The times at which particles reach each model 
level are then identified and the cloud properties at those times are recorded. 
(Dots show the time each particle reaches each model level and colours
show the cloud's vertical velocity. Only half the model levels have been 
plotted for clarity.)  c) The properties encountered by the particles at a 
given height are then arranged by the time each particle was released, forming 
a set of pseudo-time series at each height. (Dotted lines show the total 
specific water values of the cloud at the time each particle reached a given 
height.  The 600 m, 800 m, 1000 m, 1200 m, and 1400 m height particle values 
are highlighted and labeled.  Only half the model levels have been plotted.)  
d) Correlations are then taken between the properties of the particles at 
release and the properties at higher levels to calculate correlation profiles. 
(Solid line shows the correlation between total specific humidity of the 
particles at release and the total specific humidity of the cloud at various
heights. Dotted lines show the 99\% confidence level for a correlation to be
significantly different than zero.)''

\vspace{5mm}

%-----------------------------------

\textbf{Section 4.1. How do these results compare to the results in the 
``Nature and Nurture'' paper by Romps and Kuang?}

\vspace{5mm}

We compare our results with Romps and Kuang 2011 in the discussion in Section 
4.3.  However, it does make sense to address this Section 4.1, so we have added 
the following paragraph at the end of Section 4.1:

``Our results largely agree with the results of Romps and Kuang [2010]: 
upper-level cloud properties are governed by the entrainment and detrainment
experienced by the cloud as it rises, and cloud base properties have little 
influence on upper-level cloud properties, suggesting that nurture is more 
important than nature in determining shallow cumulus cloud properties.  The 
exception to this is cloud area, which is correlated with cloud base area and 
which Romps and Kuang were not able to examine using their parcel model.
Nevertheless, cloud base area and entrainment/detrainment rates still exert 
roughly equal influence over upper-level cloud area.''

\vspace{5mm}

%-----------------------------------

\textbf{Section 4.2. This method for calculating the cloud-top properties 
relies on collecting statistics on the grid cell(s) that first contains liquid 
water at a given height. Is this not prone to large numerical error? How can we 
be sure that the results from this method can be trusted to give meaningful
statistics? A more robust method would be to average properties some distance 
(say, 100 m) below the cloud top and the same distance above the cloud top.}

\vspace{5mm}

We have redone these calculations using environmental properties in an 125m 
region centred on the cloud top (5 grid cells in the vertical), and cloud 
properties over the top 100 m of the cloud (4 grid cells in the vertical).  
This does appear to remove noise from the calculation, since while the results 
are not significantly different, the p-values of the results are much higher.

The main differences between the calculations using the immediate cloud-top and
the top 100 m are greater differences in the cloud properties between the tall
and short clouds, and less differences in the environment properties between 
the tall and short clouds.  The environmental vertical velocity between 
550-750 m still shows a weakend effect on future cloud height, but cloud top 
height now appears entirely insensitive to 750-1000 m environmental properties.

We have updated the values in Table 1 with these calculations, and changed 
the text in Section 4.2 to reflect the new calculation method, and the slightly 
modified results. 

\vspace{5mm}

%-----------------------------------

\textbf{p. 23251, line 7. ``but not buoyancy''? Why would the upward velocity 
of parcels in a convective boundary layer not be controlled by buoyancy?}

\vspace{5mm}

This was a poor choice of words.  First, we should have said ``the upward 
velocity of plumes'', not ``air parcels''. Second, we intended to say that the 
mean vertical velocity of sub-cloud plumes is not correlated with the plume's 
mean buoyancy.  We have changed this sentence to read: ``Conversely, the mean 
upward velocity of plumes in this region is uncorrelated with the plume's mean 
buoyancy and plume velocity anomalies dissipate quickly as the plumes rise,
suggesting the sub-cloud plume dynamics are dominated by inertia and pressure
effects.''

\vspace{5mm}

%-----------------------------------

\textbf{p. 23251, lines 17-18. These lines state: ``the fate of clouds... is
determined by a race btw. the rate the cloud moves upward and rate the cloud is 
mixed away.'' This is reminiscent of Neggers et al, ``A multiparcel model...'' 
Is there support for the Neggers et al theory in this paper? Please elaborate. 
The discussion on p. 23245 lines 24-28 seems to suggest the opposite.}

\vspace{5mm}

The assumption made by Neggers et al. that cloud parcels enter cloud base with 
a wide range of properties is contradicted by our findings, as we state on 
p. 23245 lines 24-28.  We have not attempted to examine the assumption made by 
Neggers et al. that fractional entrainment rate is inversely proportional to 
vertical velocity, although we agree that such an assumption would be 
consistent with our findings.  However, we would say that there is are much
stronger correlations between the height a cloud achieves and its area than 
its vertical velocity.  This would lead us to speculate that the inverse 
proportionality between fractional entrainment rate and vertical velocity 
assumed by Neggers et al. is actually caused by larger clouds protecting their 
core more effectively. Parcels in clouds with large areas are more protected
from entrainment events, which allows them to achieve high buoyancies and 
vertical velocities.  We have added the following paragraph to the discussion.

``Neggers et al. [2002] construct a theory in which fractional entrainment rate 
is inversely proportional to vertical velocity and cloud parcels enter cloud 
base with a wide range of properties.  The variations in parcel properties then 
set the entrainment rate and thus control the future evolution of the parcel 
properties.  Our results disagree with the assumption of Neggers et al. that 
parcels have a range of initial thermodynamic conditions.  We have not directly 
examined the dependance of entrainment rate on vertical velocity; however, the 
strong relationship between eventual height reached by the clouds and the cloud 
area suggests that fractional entrainment is more likely dependent on cloud 
area and any relationship between vertical velocity and entrainment is due to 
larger area clouds shielding their cores from entrainment, producing higher 
buoyancies and vertical velocities.''

\vspace{5mm}

\end{document}
