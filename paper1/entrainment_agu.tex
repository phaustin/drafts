%%%%%%%%%%%%%%%%%%%%%%%%%%%%%%%%%%%%%%%%%%%%%%%%%%%%%%%%%%%%%%%%%%%%%%%%%%%%
% AGUtmpl.tex: this template file is for articles formatted with LaTeX2e,
% Modified March 2009
%
% This template includes commands and instructions
% given in the order necessary to produce a final output that will
% satisfy AGU requirements.
%
% PLEASE DO NOT USE YOUR OWN MACROS
%
% For more information on using the AGUTeX macro package,
% see agudocs.tex or agudocs.pdf
%
%%%%%%%%%%%%%%%%%%%%%%%%%%%%%%%%%%%%%%%%%%%%%%%%%%%%%%%%%%%%%%%%%%%%%%%%%%%%
%
% All questions should be e-mailed to author.help@agu.org.
%
%%%%%%%%%%%%%%%%%%%%%%%%%%%%%%%%%%%%%%%%%%%%%%%%%%%%%%%%%%%%%%%%%%%%%%%%%%%%
%
% Step 1: set the \documentclass
%
% The three options for article format are: two-column (default),
% draft, for initial article submission; and galley for narrow
% single columns.
%
% PLEASE USE THE DRAFT OPTION TO SUBMIT YOUR PAPERS
% The draft option produces double spaced output
%
% Choose the journal abbreviation for the journal you are
% submitting to:

% jgrga JOURNAL OF GEOPHYSICAL RESEARCH
% gbc   GLOBAL BIOCHEMICAL CYCLES
% grl   GEOPHYSICAL RESEARCH LETTERS
% pal   PALEOCEANOGRAPHY
% ras   RADIO SCIENCE
% rog   REVIEWS OF GEOPHYSICS
% tec   TECTONICS
% wrr   WATER RESOURCES RESEARCH
% gc    GEOCHEMISTRY, GEOPHYSICS, GEOSYSTEMS

% (If you are submitting to a journal other than jgrga,
% substitute the initials of the journal for "jgrga" below)

\documentclass[draft,jgrga]{agutex}

%%%%%%%%%%%%%%%%%%%%%%%%%%%%%%%%%%%%%%%%%%%%%%%%%%%%%%%%%%
%%%% optional article formats author might want to use

% To produce a galley version:
%\documentclass[galley,jgrga]{AGUTeX}

% To produce a two columned version:
% \documentclass[jgrga]{AGUTeX}

%%%%%%%%%%%%%%%%%%%%%%%%%%%%%%%%%%%%%%%%%%%%%%%%%%%%%%%%%%%%%%%%%%%%%%%%%
% OPTIONAL:
% To print your article using PostScript fonts, uncomment this:
% \usepackage{agu-ps}
% You many need to edit the top of agu-ps to use the names of the PS
% fonts on your system.

%%%%%%%%%%%%%%%%%%%%%%%%%%%%%%%%%%%%%%%%%%%%%%%%%%%%%%%%%%%%%%%%%%%%%%%%%
% OPTIONAL:
% To Create numbered lines:

% If you don't already have lineno.sty, you can download it from
% http://www.ctan.org/tex-archive/macros/latex/contrib/ednotes/
% (or google lineno.sty ctan), available at TeX Archive Network (CTAN).
% Take care that you always use the latest version.

% To activate the commands, uncomment \usepackage{lineno}
% and \linenumbers*[1]command, below:

\usepackage{lineno}
\linenumbers*[1]

% Author names in capital letters:
\authorrunninghead{DAWE ET AL.}

% Shorter version of title entered in capital letters:
\titlerunninghead{Entrainment}

% Author mailing address: please repeat this command for
% each author and alphabetize authors:

%\authoraddr{R. C. Bales,
%Department of Hydrology and Water Resources, University of
%Arizona, Harshbarger Building 11, Tucson, AZ 85721, USA.
%(roger@hwr.arizona.edu)}

%\authoraddr{J. R. McConnell, Division of Hydrologic
%Sciences, 123 Main Street, Desert Research Institute, Reno, NV
%89512, USA.}

%\authoraddr{E. Mosley-Thompson, Department of Geography,
%Ohio State University, 123 Orange Boulevard, Columbus, OH 43210,
%USA.}

%\authoraddr{R. Williams, Department of Space Sciences, University of
%Michigan, 123 Brown Avenue, Ann Arbor, MI 48109, USA.}

\begin{document}

%% ------------------------------------------------------------------------ %%
%
%  TITLE
%
%% ------------------------------------------------------------------------ %%

\title{Interpolation of LES cloud surfaces for use in direct calculations
of entrainment and detrainment}

% e.g., \title{Terrestrial ring current:
% Origin, formation, and decay $\alpha\beta\Gamma\Delta$}
% You may use \\ to break the title over several lines.

%% ------------------------------------------------------------------------ %%
%
%  AUTHORS AND AFFILIATIONS
%
%% ------------------------------------------------------------------------ %%


%Use \author{\altaffilmark{}} and \altaffiltext{}

% \altaffilmark will produce footnote;
% matching altaffiltext will appear at bottom of page.
% May use \\ to start a new line.

%\authors{R. C. Bales, \altaffilmark{1}
%E. Mosley-Thompson, \altaffilmark{2} R. Williams, \altaffilmark{3}
%and J. R. McConnell\altaffilmark{4}}

%\altaffiltext{1}{Department of Hydrology and Water Resources,
%University of Arizona, Tucson, Arizona, USA.}

%\altaffiltext{2}{Department of Geography, Ohio State University,
%Columbus, Ohio, USA.}

%\altaffiltext{3}{Department of Space Sciences, University of
%Michigan, Ann Arbor, Michigan, USA.}

%\altaffiltext{4}{Division of Hydrologic Sciences, Desert Research
%Institute, Reno, Nevada, USA.}

%% ------------------------------------------------------------------------ %%
%
%  ABSTRACT
%
%% ------------------------------------------------------------------------ %%

% >> Do NOT include any \begin...\end commands within
% >> the body of the abstract.

\begin{abstract}
Direct calculations of the entrainment and detrianment of air between
clouds and their environment require a knowledge of the relative velocity
difference between the air and the cloud surface. LES model grids
force the distance moved by the cloud surface over a timestep to be
either zero or the width of a model grid cell, while in reality the
cloud surface tends to move at a more constant rate. Here we present
a method for the subgrid interpolation of a cloud surface given total
humidity and saturated humidity on a regular LES grid. This method
is used to calculate entrainment and detrainment rates for an LES
model, which are compared with rates inferred from bulk conserved
tracer calculations. 
\end{abstract}

%% ------------------------------------------------------------------------ %%
%
%  BEGIN ARTICLE
%
%% ------------------------------------------------------------------------ %%

% The body of the article must start with a \begin{article} command
%
% \end{article} must follow the references section, before the figures
%  and tables.

\begin{article}

%% ------------------------------------------------------------------------ %%
%
%  TEXT
%
%% ------------------------------------------------------------------------ %%
\section{Introduction}

The largest uncertainties in Global Circulation Model (GCM) simulations
come from the subgridscale parameterization of clouds. The IPCC 2007
report found the *TK* models they examined showed a *TK* W
m$^{-2}$ range in the cloud feedback response, relative to a *TK*
W m$^{-2}$ total climate sensitivity.  Improvements in the accuracy of 
these subgridscale cloud parametrizations are neccessary to simulate 
the rate and spatial pattern of global warming.

Proper simulation of the subgridscale effect of cumulus clouds in
GCMs requires understanding the rates at which air is entrained into
and detrained from the clouds. Cloud entrainment and detrainment rates
exert influences on profiles of cloud properties, the height of the
cloud tops, the amount of heat and moisture the clouds transport upwards,
and the heights at which the clouds deposit that heat and moisture.
They also have effects on the vertical transport of aerosols out of
the boundary layer and the rate at which chemical reactions can occur
in those aerosols. Several approaches to the parametrization of entrainment
and detrainment rates have been proposed, including *TK*.

Large Eddy Simulation (LES) is an important tool used in the study
of cloud entrainment and detrainment. LES models achieve grid resolutions
on the order of 10-100 m, so that the smallest length scales resolved
by the model touch the small wave number end of the Kolmagorov -5/3
turbulence spectrum. This allows for a relatively simple turbulence
model that captures the important statistics of the subgridscale eddy
fluxes and thus, an accurate representation of the atmospheric physics
of a domain \textasciitilde{}10 km$^{2}$, which is enough to simulate
a field of clouds. LES simulations can be ground-truthed against results
taken from large field surveys, such as the Barbados Oceanographic
and Meteorological Experiment (BOMEX) or the Atmospheric Radiation
Measurement (ARM) Program, and such comparisons show good agreement
between the LES simulations and field data.

Several recent studies have looked at the lifecycle of individual
clouds taken from LES models, trying to break the cloud field into
its component parts. Estimates of entrainment and detrainment rates
for individual clouds would be quite useful in these types of studies,
but are difficult to achieve. Entrainment and detrainment rates are 
typically calculated in LES simulations by recording budgets
of bulk conserved tracer variables, such as the total humidity or
the liquid water moist static energy, and inferring the amount of
fluid exchange between cloud and clear air that is needed to explain
the rate at which that tracer is being vertically advected within
the cloud field. These budgets typically assume the clouds and the
cloud environment are horizontally homogenous slabs; this is a much
less accurate assumption on the level of an individual cloud.

Alternatively, entrianment and detrianment could simply be calculated
directly from the LES velocity and humidity field.  \cite{Siebesma1998} 
defines Entrainment and Detrainment as

\begin{equation}
E = -\frac{1}{A}\oint_{\mathbf{\hat{n}}\cdot(\mathbf{u} - \mathbf{u_i}) < 0}
\mathbf{\hat{n}}\cdot(\mathbf{u}-\mathbf{u_i})dl
\end{equation}
\begin{equation}
D = \frac{1}{A}\oint_{\mathbf{\hat{n}}\cdot(\mathbf{u} - \mathbf{u_i}) > 0}
\mathbf{\hat{n}}\cdot(\mathbf{u}-\mathbf{u_i})dl
\end{equation}

where $E$ and $D$ are the entrainment and detrainment rates in kg 
m$^{-3}$ s$^{-1}$, u is velocity in m s$^{-1}$, *TK*.  However, the
accuracy of this method suffers from the need to calculate the velocity 
of the air relative to the cloud surface.  In reality 
these velocities are very nearly identical, \textasciitilde{} 1 m 
s$^{-1}$, but the discrete nature of the LES model grid forces the 
modeled surface velocity to be either 0 m s$^{-1}$ or $\Delta x / \Delta t 
\approx$ 30 m s$^{-1}$, where $\Delta x$ is the model grid spacing and 
$\Delta t$ is the model timestep.  The surface of the cloud only moves 
when a grid cell's humidity reaches saturation, and when it 
does, an entire grid cell worth of fluid leaves or enters the cloud.  This 
causes both the entrainment and detrainment to be over-estimated.

Here we present a method for calculation of the cloud entrainment and 
detrainment rates that relies on interpolation of the subgrid location 
of the cloud surface.  In section 2 we describe this method, in section 
3 we compare this calculation with entrainment and detrainment rates 
calculated using bulk conserved tracer budgets, and in section 4 we 
discuss our results.  With this method, we believe accurate estimates 
of the cloud entrainment and detrainment rates are possible for 
individual LES clouds.

\section{Method}

\subsection{Flux Calculations}

Consider a discrete 3-d numerical model operating on an Arakawa C-grid
(Fig \ref{fig:C_grid}) containing a cloud surface $\mathbf{A}$, where
the vectors representing $\mathbf{A}$ point outward from the cloud.  For 
the moment we neglect specifying the interpolation scheme that determines 
this surface.  This surface, combined with the walls of the grid cell, 
encloses a cloud volume $V$.  

Converted to our notation, \cite{Siebesma1998} gives the net Entrainment 
and Detrainment over the cloud interface to be:

\begin{equation}
\label{eq:E_minus_D} 
E - D = \int_A \rho ( \mathbf{u} -  \mathbf{u_i}) \cdot d\mathbf{A}.
\end{equation}

where $\rho$ is the air density in kg m$^{-3}$, $\mathbf{u}$ is the velocity
of the air in m s$^{-1}$ and $\mathbf{u_i}$ is the velocity of the cloud interface. 
Calculating this integral for a numerical model cloud field would require interpolation
of the velocity field to the surface of  $\mathbf{A}$ and a record of the time 
evoluion of $\mathbf{A}$.  Instead, we seek a simplified but equivalent calculation.

To calculate the velocity of the cloud interface, we make use of the Leibnitz Theorem:

\begin{equation}
\label{eq:leibnitz} 
\frac{d}{dt}\int_{V(t)} \rho dV = \int_{V(t)} \frac{\partial \rho}{ \partial t} dV 
                                + \int_{A(t)} \rho \mathbf{u_i}\cdot d\mathbf{A}
\end{equation}

Since the walls of the grid cell do not move, if we assume
${\partial \rho}/{ \partial t} \approx 0$ we can combine equations (\ref{eq:E_minus_D}) 
and (\ref{eq:leibnitz}) to give:

\begin{equation}
E - D = \rho \int_A \mathbf{u} \cdot d\mathbf{A} -  \rho \frac{d}{dt}\int_{V(t)} dV.
\end{equation}

Next we apply the divergence theorem to simplify the flux integral through $\mathbf{A}$:

\begin{equation}
\label{eq:divergence} 
\int_{V} \nabla \cdot (\rho \mathbf{u}) dV = \int_{A} \rho \mathbf{u}\cdot d\mathbf{A}
\end{equation}

Due to mass conservation, $\nabla \cdot (\rho \mathbf{u}) = 0$, which implies that
the massflux passing through $\mathbf{A}$ is equal to the massflux entering the volume $V$ 
through the walls of the model grid cell.  If we take the u and v velocities at the 
grid cell walls to be constant, we can calculate these fluxes simply by multiplying the 
velocity by the area of the grid cell wall occupied by cloud.  Thus:

\begin{equation}
E - D = \rho \int_S \mathbf{u} \cdot d\mathbf{S} -  \rho \frac{dV}{dt}
\end{equation}.

Thus, to calculate the entrainment and detrainment, we need to calculate the flux through 
the cloudy portion of the grid cell walls, and the rate of change of the cloud volume 
inside the grid cell.

\subsection{Cloud Surface Interpolation}

There are a multitude of interpolation schemes that can be used to determine the cloud 
volume and cell wall area in a numerical model.  The simplest would be to assume 
that saturated grid cells are completely filled with cloud and unsaturated grid cells 
have no cloud at all.  We refer to this as the "no inerpolation" case.

At the other end of the range of interpolation schemes, we can use linear (or higher order) 
interpolation to estimate the surface where the total water equals the saturated humidity 
and calculate the cloud volume and cell surface areas at each timestep.  Several standard 
techniques exist for this kind of calculation in the field of computer visualization, such
as the Marching Squares algorithm *TK*.  These techniques, however, are generally complex and 
require large amounts of calculation to perform.

Instead of these complex interpolation schemes, we propose a compromise scheme that calculates 
an approximate area and volume for the cloud surface while requiring much less computation time.
To do this, we divide each grid cell into six pyramids (Figure \ref{fig:pyramid_scheme}).  Next,
we assign the center of each grid cell to be cloudy or clear based upon the specific humidity.


\section{Comparison with bulk conserved tracer calculations}


\section{Discussion}


%%% End of body of article:

%  ACKNOWLEDGMENTS

\begin{acknowledgments}
Figures were generated using the matplotlib library in the Python
programming language.
\end{acknowledgments}



%% ------------------------------------------------------------------------ %%
%
%  REFERENCE LIST AND TEXT CITATIONS
%
% Either type in your references using
% \begin{thebibliography}{}
% \bibitem{}
% Text
% \end{thebibliography}
%
% Or,
%
% If you use BiBTeX for your References, please produce your .bbl
% file and copy the contents into your paper here.
%
% Follow these steps:
% 1. Run LaTeX on your LaTeX file.
%
% 2. Run BiBTeX on your LaTeX file.
%
% 3. Open the new .bbl file containing the reference list and
%   copy all the contents into your LaTeX file here.
%
% 4. Comment out the old \bibliographystyle and \bibliography commands.
%
% 5. Run LaTeX on your new file before submitting.
%
% AGU does not want a .bib or a .bbl file, but asks that you
% copy in the contents of your .bbl file here.

\bibliographystyle{agu}
\bibliography{entrainment_interpolation}{} 

%\bibitem[{\textit{Kilby}(2008)}]{jskilby}
%Kilby, J. S. (2008), Invention of the integrated circuit, {\it IEEE
%Trans. Electron Devices,} \textit{23}, 648--650.

%\bibitem[{\textit{Kilby et al.}(2008)}]{jskilbye}
%Kilby, J. S., S. Smith, and R. Jones (2008), Invention of the
%integrated circuit, {\it IEEE Trans. Electron Devices,} \textit{23},
%648--650.

%Reference citation examples:

%...as shown by \textit{Kilby} [2008].
%...has been shown [\textit{Kilby et al.}, 2008].

%...as shown by \cite{jskilby}.
%...has been shown \citep{jskilbye}.


%% ------------------------------------------------------------------------ %%
%
%  END ARTICLE
%
%% ------------------------------------------------------------------------ %%

\end{article}

%% Enter Figures and Tables here:

% When submitting articles through the GEMS system:
% COMMENT OUT ANY COMMANDS THAT INCLUDE GRAPHICS.
-
% Figure captions go below this illustration; Table captions go above tables

% ONE-COLUMN figure/table, including eps graphics
%
% \begin{figure}
% \noindent\includegraphics[width=20pc]{samplefigure.eps}
% \caption{Caption text here}
% \end{figure}
% \end{document}
%
% \begin{table}
% \caption{}
% \end{table}
%
% ---------------
% TWO-COLUMN figure/table
%
% \begin{figure*}
% \noindent\includegraphics[width=39pc]{samplefigure.eps}
% \caption{Caption text here}
% \end{figure*}
%
% \begin{table*}
% \caption{Caption text here}
% \end{table*}
%
% see below for how to make landscape figures or tables

%%% End the article here:

\end{document}

%%%%%%%%%%%%%%%%%%%%%%%%%%%%%%%%%%%%%%%%%%%%%%%%%%%%%%%%%%%%%%%
