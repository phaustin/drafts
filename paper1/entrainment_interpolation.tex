%% LyX 1.6.2 created this file.  For more info, see http://www.lyx.org/.
%% Do not edit unless you really know what you are doing.
\documentclass[english]{article}
\usepackage[T1]{fontenc}
\usepackage[latin9]{inputenc}

%%%%%%%%%%%%%%%%%%%%%%%%%%%%%% User specified LaTeX commands.

\author{Jordan T Dawe, Phil Austin}


\usepackage{babel}

\begin{document}

\title{Interpolation of LES cloud surfaces for use in direct calculations
of entrainment and detrainment}
\maketitle
\begin{abstract}
Direct calculations of the entrainment and detrianment of air between
clouds and their environment require a knowledge of the relative velocity
difference between the air and the cloud surface. LES model grids
force the distance moved by the cloud surface over a timestep to be
either zero or the width of a model grid cell, while in reality the
cloud surface tends to move at a more constant rate. Here we present
a method for the subgrid interpolation of a cloud surface given total
humidity and saturated humidity on a regular LES grid. This method
is used to calculate entrainment and detrainment rates for an LES
model, which are compared with rates inferred from bulk conserved
tracer calculations. 
\end{abstract}

\section{Introduction}

The largest uncertainties in Global Circulation Model (GCM) simulations
come from the subgridscale parameterization of clouds. The IPCC 2007
report found the *TK* models they examined showed a *TK* W
m$^{-2}$ range in the cloud feedback response, relative to a *TK*
W m$^{-2}$ total climate sensitivity.  Improvements in the accuracy of 
these subgridscale cloud parametrizations are neccessary to simulate 
the rate and spatial pattern of global warming.

Proper simulation of the subgridscale effect of cumulus clouds in
GCMs requires understanding the rates at which air is entrained into
and detrained from the clouds. Cloud entrainment and detrainment rates
exert influences on profiles of cloud properties, the height of the
cloud tops, the amount of heat and moisture the clouds transport upwards,
and the heights at which the clouds deposit that heat and moisture.
They also have effects on the vertical transport of aerosols out of
the boundary layer and the rate at which chemical reactions can occur
in those aerosols. Several approaches to the parametrization of entrainment
and detrainment rates have been proposed, including *TK*.

Large Eddy Simulation (LES) is an important tool used in the study
of cloud entrainment and detrainment. LES models achieve grid resolutions
on the order of 10-100 m, so that the smallest length scales resolved
by the model touch the small wave number end of the Kolmagorov -5/3
turbulence spectrum. This allows for a relatively simple turbulence
model that captures the important statistics of the subgridscale eddy
fluxes and thus, an accurate representation of the atmospheric physics
of a domain \textasciitilde{}10 km$^{2}$, which is enough to simulate
a field of clouds. LES simulations can be ground-truthed against results
taken from large field surveys, such as the Barbados Oceanographic
and Meteorological Experiment (BOMEX) or the Atmospheric Radiation
Measurement (ARM) Program, and such comparisons show good agreement
between the LES simulations and field data.

Several recent studies have looked at the lifecycle of individual
clouds taken from LES models, trying to break the cloud field into
its component parts. Estimates of entrainment and detrainment rates
for individual clouds would be quite useful in these types of studies,
but are difficult to achieve. Entrainment and detrainment rates are 
typically calculated in LES simulations by recording budgets
of bulk conserved tracer variables, such as the total humidity or
the liquid water moist static energy, and inferring the amount of
fluid exchange between cloud and clear air that is needed to explain
the rate at which that tracer is being vertically advected within
the cloud field. These budgets typically assume the clouds and the
cloud environment are horizontally homogenous slabs; this is a much
less accurate assumption on the level of an individual cloud.

Alternatively, entrianment and detrianment could simply be calculated
directly from the LES velocity and humidity field.  \cite{Siebesma1998} 
defines Entrainment and Detrainment as

\begin{equation}
E = -\frac{1}{A}\oint_{\mathbf{\hat{n}}\cdot(\mathbf{u} - \mathbf{u_i}) < 0}
\mathbf{\hat{n}}\cdot(\mathbf{u}-\mathbf{u_i})dl
\end{equation}
\begin{equation}
D = \frac{1}{A}\oint_{\mathbf{\hat{n}}\cdot(\mathbf{u} - \mathbf{u_i}) > 0}
\mathbf{\hat{n}}\cdot(\mathbf{u}-\mathbf{u_i})dl
\end{equation}

where $E$ and $D$ are the entrainment and detrainment rates in kg 
m$^{-3}$ s$^{-1}$, u is velocity in m s$^{-1}$, *TK*.  However, the
accuracy of this method suffers from the need to calculate the velocity 
of the air relative to the cloud surface.  In reality 
these velocities are very nearly identical, \textasciitilde{} 1 m 
s$^{-1}$, but the discrete nature of the LES model grid forces the 
modeled surface velocity to be either 0 m s$^{-1}$ or $\Delta x / \Delta t 
\approx$ 30 m s$^{-1}$, where $\Delta x$ is the model grid spacing and 
$\Delta t$ is the model timestep.  The surface of the cloud only moves 
when a grid cell's humidity reaches saturation, and when it 
does, an entire grid cell worth of fluid leaves or enters the cloud.  This 
causes both the entrainment and detrainment to be over-estimated.

Here we present a method for calculation of the cloud entrainment and 
detrainment rates that relies on interpolation of the subgrid location 
of the cloud surface.  In section 2 we describe this method, in section 
3 we compare this calculation with entrainment and detrainment rates 
calculated using bulk conserved tracer budgets, and in section 4 we 
discuss our results.  With this method, we believe accurate estimates 
of the cloud entrainment and detrainment rates are possible for 
individual LES clouds.

\section{Method}

In this section we present a general method for calculation of entrainment and 
detrainment through a cloud surface in a discrete numerical model.

\cite{Siebesma1998} gives the net Entrainment and Detrainment over the 
cloud interface to be:

\begin{equation}
\label{eq:E_minus_D} 
E - D = \int_A \rho ( \mathbf{u} -  \mathbf{u_i}) \cdot d\mathbf{A}.
\end{equation}

Consider a discrete 3-d numerical model operating on an Arakawa C-grid
(Fig \ref{fig:C_grid}). At any timestep $t = n\Delta t$, the humidity
of each grid



Leibnitz theorem:

\begin{equation}
\label{eq:leibnitz} 
\frac{d}{dt}\int_{V(t)} \rho dV = \int_{V(t)} \frac{\partial \rho}{ \partial t} dV 
                                + \int_{A(t)} \rho \mathbf{u_i}\cdot d\mathbf{A}
\end{equation}

Since ${\partial \rho}/{ \partial t} \approx 0$ we can combine equations (\ref{eq:E_minus_D})
and (\ref{eq:leibnitz}) to give:

\begin{equation}
E - D = \int_A \rho \mathbf{u} \cdot d\mathbf{A} - \frac{d}{dt}\int_{V(t)} \rho dV.
\end{equation}

Divergence theorem:

\begin{equation}
\label{eq:divergence} 
\int_{V(t)} \nabla \cdot (\rho \mathbf{u}) dV = \int_{A(t)} \rho \mathbf{u}\cdot d\mathbf{A}
\end{equation}



\section{Comparison with bulk conserved tracer calculations}


\section{Discussion}

Acknowledgments

Figures were generated using the matplotlib library in the Python
programming language.

\bibliographystyle{agu}

\bibliography{entrainment_interpolation}{} 

\begin{figure}
\label{fig:C_grid}
\caption{Blah blah blah}
\end{figure}

\end{document}
